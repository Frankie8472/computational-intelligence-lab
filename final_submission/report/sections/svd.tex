\subsubsection{\ac{SVD}}\label{sec:svd}

Collaborative filtering using \ac{SVD} takes advantage of feature reduction. It takes a matrix $A\in \mathbb{R}^{m \times n}$ and decomposes it into three matrices $U\in \mathbb{R}^{m \times m}\text{, }\Sigma\in\mathbb{R}^{m\times n}\text{, }V\in\mathbb{R}^{n\times n}$, so that $A = U\Sigma V$. Using dimension reduction on $\Sigma$, one can approximate $A$ by keeping $k\leq\text{\textit{rank(A)}}$ latent factors. In our \ac{SVD} baseline algorithm, we replaced missing ratings in $A$ with the adjusted item mean. That is, for a missing rating $r\in[1, 5]$ of user $u$ and item $i$, initialize $r$ to be the mean of all known ratings for $i$. If user $u$ tends to give rather low or high ratings, i.e., the mean of all known ratings of $u$ is less than 2 or greater than 4, decrease or increase $r$ by 1. Further, we have found that a $k\in [8,17]$ lead to the best results.